---
---
\tikzset{
    hyperlink node/.style={
        alias=sourcenode,
        append after command={
            let \p1 = (sourcenode.north west),
                \p2=(sourcenode.south east),
                \n1={\x2-\x1},
                \n2={\y1-\y2} in
            node [inner sep=0pt, outer sep=0pt,anchor=north west,at=(\p1)] {\href{#1}{\XeTeXLinkBox{\phantom{\rule{\n1}{\n2}}}}}
                    %xelatex needs \XeTeXLinkBox, won't create a link unless it
                    %finds text --- rules don't work without \XeTeXLinkBox.
                    %Still builds correctly with pdflatex and lualatex
        }
    }
}

\definecolor{csource}{RGB}{222,168,167}
\definecolor{ctarget}{RGB}{222,168,167}
\definecolor{cjava}{RGB}{156,186,95}
\definecolor{ccoqp}{RGB}{222,168,167}
\definecolor{ccoqc}{RGB}{161,187,215}

\tikzstyle{coqp}=[
 draw=ccoqp!150,fill=ccoqp
]

\tikzstyle{coqc}=[
 draw=ccoqc!150,fill=ccoqc
]

\tikzstyle{java}=[
 draw=cjava!150,fill=cjava
]

\tikzstyle{java-api}=[
 draw=cjava,fill=cjava!50,
 minimum width=21em,
 minimum height=1.4em,
 xshift=5em,
 font=\scriptsize\sffamily,
]

\tikzstyle{source}=[
 draw=csource!150,fill=csource
]

\tikzstyle{target}=[
 draw=ctarget!150,fill=ctarget
]

\tikzstyle{lang}=[
 thick,
 minimum width=3em,
 minimum height=1.4em,
 font=\scriptsize\sffamily,
]

\tikzstyle{sep-caption}=[
 thick,
 font=\scriptsize\sffamily,
]

\tikzstyle{tcoqp}=[
 draw=ccoqp!150
]

\tikzstyle{tcoqc}=[
 draw=ccoqc!150
]

\tikzstyle{trans}=[
 draw,semithick,-latex,
 font=\scriptsize\sffamily,
]

\tikzstyle{separation}=[
 draw,very thick, gray
]

\newcommand{\tikzoptim}[3][]{
   \path[trans, draw=#1!150, fill=#1!150] (#2) edge [looseness=10, loop above, draw=#1!150]
       node [pos=0.5](#2-optim) {}
       node [yshift = 0.5em,below] {\href{#3}{\phantom{XXX}} \\ \href{#3}{\phantom{XXX}}} ();
}
\newcommand{\tikzloop}[4][]{
   \path[trans, draw=#1!150, fill=#1!150] (#2) edge [looseness=10, loop above, draw=#1!150]
       node [pos=0.5](#2-optim) {#4}
       node [yshift = 0.5em,below] {\href{#3}{\phantom{XXX}} \\ \href{#3}{\phantom{XXX}}} ();
}
\newcommand{\tikzoptimdashed}[3][]{
   \path[trans, draw=#1!150, fill=#1!150] (#2) edge [looseness=10, loop above, dashed, draw=#1!150]
       node [pos=0.5](#2-optim) {}
       node [yshift = 0.5em,below] {\href{#3}{\phantom{XXX}} \\ \href{#3}{\phantom{XXX}}} ();
}

\newcommand{\transref}[2]{
   \foreach \i in {0,...,#1} { node [pos=\i/#1, circle] {\href{#2}{\phantom{X}}}}
}

\newcommand{\transocamljava}[1]{
 \path[draw,thick,latex-latex, densely dashed,
      shorten >=0.2em, shorten <=0.2em,
       java] ([xshift=-0.75em] #1.south) -- ([xshift=-0.75em] #1-java.north);
}

\makeatletter
\tikzset{
  box/.style={
    minimum height=0.8cm,
    draw,
    rounded corners,
    rectangle
  },
  moveable rectangle/condition/.code={%
    \pgfmathparse{(\mrwidth > 1cm ? "(a.north east)" : "(a.south west)")}
    \let\mrvector=\pgfmathresult
  },
  moveable/.is if=mr@moveable,
}
\def\mrvector{(0,0)}
\newif\ifmr@moveable
\pgfdeclareshape{moveable rectangle}{%
  \inheritsavedanchors[from=rectangle]%
  \anchor{north west}{%
    \ifmr@moveable
    \northeast
    \pgf@xa=\pgf@x
    \southwest
    \advance\pgf@xa by -\pgf@x
    \edef\mrwidth{\the\pgf@xa}%
    \pgfkeys{/tikz/moveable rectangle/condition}%
    \tikz@scan@one@point\pgfutil@firstofone\mrvector
    \pgf@xa=-\pgf@x
    \pgf@ya=-\pgf@y
    \northeast
    \advance\pgf@ya by \pgf@y
    \southwest
    \advance\pgf@xa by \pgf@x
    \pgf@x=\pgf@xa
    \pgf@y=\pgf@ya
    \else
    \northeast
    \pgf@ya=\pgf@y
    \southwest
    \pgf@y=\pgf@ya
    \fi
  }
  \inheritbackgroundpath[from=rectangle]
}
\makeatother
